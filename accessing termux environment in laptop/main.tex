\documentclass[conference]{IEEEtran}
\usepackage{amsmath}
\usepackage{verbatim}
\usepackage{hyperref}
\usepackage{graphicx}
\usepackage{enumitem}
\usepackage{titlesec}
\usepackage{color}

\title{\textbf{Comprehensive Guide to SSH Access: Termux and Laptop/PC Integration}}
\author{\IEEEauthorblockN{Errala Paulsonashish}
\IEEEauthorblockA{Email: paulsonashisherrala24@gmail.com}}

\begin{document}

\maketitle

\begin{abstract}
This guide provides step-by-step instructions for setting up SSH connections between a Termux environment on an Android device and a laptop or PC. It covers both accessing Termux remotely from a laptop and using Termux to access a laptop, along with troubleshooting and security considerations for seamless integration.
\end{abstract}

\section{Introduction}
SSH (Secure Shell) enables secure communication between devices, enhancing productivity and flexibility in mobile computing. This guide explains how to set up bidirectional SSH access: connecting to Termux from a laptop and accessing a laptop from Termux. These setups allow you to leverage the strengths of both devices efficiently.

\section{Prerequisites}
Ensure the following requirements are met:
\begin{itemize}[leftmargin=*]
    \item Android device with Termux installed.
    \item Laptop or PC with an SSH client (e.g., OpenSSH, PuTTY).
    \item Stable Wi-Fi connection for both devices.
    \item Basic familiarity with command-line interfaces.
\end{itemize}

\section{Part 1: Connecting to Termux from a Laptop/PC}
\subsection{Step 1: Install and Start OpenSSH on Termux}
\begin{itemize}[leftmargin=*]
    \item Update Termux packages and install OpenSSH:
    \begin{verbatim}
    pkg update && pkg install openssh
    \end{verbatim}
    \item Start the SSH server:
    \begin{verbatim}
    sshd
    \end{verbatim}
\end{itemize}

\subsection{Step 2: Set a Password for Termux}
Set a password for SSH authentication:
\begin{verbatim}
passwd
\end{verbatim}

\subsection{Step 3: Find the IP Address and Username}
Retrieve your Android device's IP address:
\begin{verbatim}
ifconfig
\end{verbatim}
Look under the \texttt{wlan0} interface for the \texttt{inet} address (e.g., \texttt{192.168.1.5}).  
Find your Termux username using:
\begin{verbatim}
whoami
\end{verbatim}

\subsection{Step 4: Connect to Termux from Laptop/PC}
Use the SSH command from your laptop/PC:
\begin{verbatim}
ssh -p 8022 your_username@your_phone_ip
\end{verbatim}
Example:  
\begin{verbatim}
ssh -p 8022 u0_a123@192.168.1.5
\end{verbatim}
Ensure both devices are connected to the same Wi-Fi network.

\section{Part 2: Accessing a Laptop/PC from Termux}
\subsection{Step 1: Set Up SSH Server on Laptop/PC}
Ensure the SSH server is installed and running on your laptop/PC:
\begin{itemize}[leftmargin=*]
    \item Linux: Install OpenSSH (\texttt{sudo apt install openssh-server}).
    \item Windows: Enable OpenSSH Server through Windows Features or install PuTTY.
\end{itemize}

\subsection{Step 2: Find Laptop/PC IP Address}
Determine the laptop/PC IP address (e.g., \texttt{192.168.1.10}):
\begin{itemize}[leftmargin=*]
    \item Linux: Use \texttt{ifconfig} or \texttt{ip addr}.
    \item Windows: Use \texttt{ipconfig}.
\end{itemize}

\subsection{Step 3: Connect from Termux to Laptop/PC}
Run the following command in Termux:
\begin{verbatim}
ssh your_laptop_username@your_laptop_ip
\end{verbatim}
Example:  
\begin{verbatim}
ssh john@192.168.1.10
\end{verbatim}

\section{Security Best Practices}
\begin{itemize}[leftmargin=*]
    \item Use strong passwords or configure SSH key-based authentication.
    \item Regularly update Termux and laptop/PC software to patch vulnerabilities:
    \begin{verbatim}
    pkg update && pkg upgrade
    \end{verbatim}
    \item Avoid public Wi-Fi networks for SSH connections unless using a VPN.
\end{itemize}

\section{Troubleshooting}
If connection issues arise:
\begin{itemize}[leftmargin=*]
    \item Verify the SSH server is running (\texttt{sshd} on Termux or SSH service on laptop).
    \item Confirm the IP addresses and usernames are correct.
    \item Check firewall settings or battery optimization settings interfering with Termux.
    \item Ensure both devices are on the same Wi-Fi network.
\end{itemize}

\section{Conclusion}
Setting up bidirectional SSH access between Termux and a laptop/PC enables efficient workflows and flexibility. By following this guide, you can securely connect and manage your devices, enhancing your mobile computing capabilities.

\section{References}
\begin{itemize}[leftmargin=*]
    \item Termux Documentation: \url{https://wiki.termux.com/wiki/Main_Page}
    \item OpenSSH Documentation: \url{https://www.openssh.com/manual.html}
\end{itemize}

\end{document}
